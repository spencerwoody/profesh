%----------------------------------------------------------------------------------------
%	PACKAGES AND OTHER DOCUMENT CONFIGURATIONS
%----------------------------------------------------------------------------------------

\documentclass{article}

%SPENCER NICE FONTS
\usepackage{times}
\usepackage{avant}
\usepackage{inconsolata}

\usepackage{fancyhdr} % Required for custom headers
\usepackage{lastpage} % Required to determine the last page for the footer
\usepackage{extramarks} % Required for headers and footers
\usepackage[usenames,dvipsnames]{color} % Required for custom colors
\usepackage{graphicx} % Required to insert images
\usepackage{listings} % Required for insertion of code
\usepackage{amsmath}  % Required for \text{} function
%\usepackage{couriernew} % Required for the courier font
\usepackage{pdfpages} %Required to embed pdf file.
\usepackage{enumerate} % Required for enumerating with letters
\usepackage{amssymb} %Required for QED symbols.
\usepackage{bbm}  %Required for indicator function.
\usepackage{float}	%Required for strict figure placement [H].

%CUSTOMIZE INDENTS:
\newcommand{\myindent}{\hspace*{1cm}}
\setlength{\parindent}{0pt}


% Margins
\topmargin=-0.45in
\evensidemargin=0in
\oddsidemargin=0in
\textwidth=6.5in
\textheight=9.0in
\headsep=0.25in

\linespread{1} % Line spacing

\setcounter{secnumdepth}{0} 
\setlength\parindent{0pt} % Removes all indentation from paragraphs

%----------------------------------------------------------------------------------------
%	CODE INCLUSION CONFIGURATION

%----------------------------------------------------------------------------------------

\begin{document}

%%%%%%%%%%%%%%%%%%%%%%%%%%%%%%%%%%%%%%%%%%%%%%%%%%%%%%%%%%%%%%%%%%%%
%%                     HEADER                                     %%
%%%%%%%%%%%%%%%%%%%%%%%%%%%%%%%%%%%%%%%%%%%%%%%%%%%%%%%%%%%%%%%%%%%%
\begin{center}
	\Large{\textbf{\textsc{Spencer Woody}}} \\
	\normalsize{\texttt{spencer.woody@utexas.edu}}\\
	+1 (832) 369-9168\\
	\texttt{spencerwoody.github.io}
\end{center}


%%%%%%%%%%%%%%%%%%%%%%%%%%%%%%%%%%%%%%%%%%%%%%%%%%%%%%%%%%%%%%%%%%%%
%%                     RESEARCH                                   %%
%%%%%%%%%%%%%%%%%%%%%%%%%%%%%%%%%%%%%%%%%%%%%%%%%%%%%%%%%%%%%%%%%%%%

\section{\textbf{\textsc{research interests}}}

I am interested in problems of post-selection inference (POSI). While
there has been a lot of work done on problems of sparse signal
identification and variable selection, post-selection is an area of
statistical research which has only recently gotten attention and is
becoming increasingly relevant in a world where data are large, and
statistical analyses are often conducted after exploring the data. So
far I have worked on POSI problems in the sparse normal means problem
and spatial signal detection, but I also hope to extend this
methodolgy to heterogeneous treat effects and variable selection. 

%%%%%%%%%%%%%%%%%%%%%%%%%%%%%%%%%%%%%%%%%%%%%%%%%%%%%%%%%%%%%%%%%%%%
%%                     EDUCATION                                  %%
%%%%%%%%%%%%%%%%%%%%%%%%%%%%%%%%%%%%%%%%%%%%%%%%%%%%%%%%%%%%%%%%%%%%
\section{\textbf{\textsc{education}}}
The University of Texas at Austin (2016 -- present)\\
PhD, Statistics. (In progress) \\ 

Duke University (2011 -- 2015) \\
BS, Economics \& Statistical Sciences (dual degree). \\
 
\textit{Summary of research to be presented} \\
In spatial modeling it is often of interest to detect ``hotspots,'' or
regions of anomolous signals. Once these signals have been detected, a
natural follow-up question is how to provide confidence regions for
signal intensity accounting for selection bias, sometimes referred to
as the ``winner's curse'' or the ``look-elsewhere'' problem. In our
work we use the selection-adjusted Bayesian inference approach from
Yekutieli (2012). The main difficulty lies in computing the
probability of the selection event occuring in order to calculate the
selection-adjusted posterior. We present an approximation to this
selection probability and then apply our method to a problem of
evolutionary biology where the objective is to find geographic regions
where a sparse set of genes are naturally selected for.



\section{\textbf{\textsc{professional memberships}}}

The American Statistical Association, since 2016.\\
The International Society of Bayesian Analysis, since 2018.

%%%%%%%%%%%%%%%%%%%%%%%%%%%%%%%%%%%%%%%%%%%%%%%%%%%%%%%%%%%%%%%%%%%% 
%%                     WORK EXPERIENCE                             %%
%%%%%%%%%%%%%%%%%%%%%%%%%%%%%%%%%%%%%%%%%%%%%%%%%%%%%%%%%%%%%%%%%%%%
\section{\textbf{\textsc{employment}}}

\textit{The University of Texas at Austin} \\
Graduate Research Assistant for Eberlin Research Group.  (2017 -- present)  \\
Graduate Assistant, Department of Women's Health, Dell Medical School
(2017 -- present) \\
Graduate Research Assistant, Professor Mike Daniels (2017) \\
Graduate Teaching Assistant, Statistics for Marketing. (2016)  \\


\textit{Integra Research Group} \\
Austin, TX \\
Data Scientist Intern (2017) \\

\textit{7-Eleven} \\
Madison, WI \\
Assistant Store Manager (2015 -- 2016)



%----------------------------------------------------------------------------------------

\end{document}